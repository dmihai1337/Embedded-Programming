\documentclass[10pt,a4paper,titlepage,oneside]{article}
\usepackage{LabProtocol}

\exercise{Exercise II}

% enter your data here
\authors{
	Vorname Nachname, Matr. Nr. 0123456 \par
	{\small e0123456@student.tuwien.ac.at} \par
}


\begin{document}

\maketitle


%████████╗ █████╗ ███████╗██╗  ██╗     ██╗
%╚══██╔══╝██╔══██╗██╔════╝██║ ██╔╝    ███║
%   ██║   ███████║███████╗█████╔╝     ╚██║
%   ██║   ██╔══██║╚════██║██╔═██╗      ██║
%   ██║   ██║  ██║███████║██║  ██╗     ██║
%   ╚═╝   ╚═╝  ╚═╝╚══════╝╚═╝  ╚═╝     ╚═╝
\Task{Space Invaders Game}

\begin{qa}{Briefly describe the architecture of your \textsf{game} module. Are there any submodules? What is their purpose? How many FSMs did you use? (approximately 6-8 sentences, you can also include figures)}

\end{qa}


%%%%%%%%%%%%%%%%%%%%%%%%%%%%%%%%%%%%%%%%%%%%%%%%%%%%%%%%%%%%%%%%%%%%%%%%%%%%%%%%


%████████╗ █████╗ ███████╗██╗  ██╗    ██████╗ 
%╚══██╔══╝██╔══██╗██╔════╝██║ ██╔╝    ╚════██╗
%   ██║   ███████║███████╗█████╔╝      █████╔╝
%   ██║   ██╔══██║╚════██║██╔═██╗     ██╔═══╝ 
%   ██║   ██║  ██║███████║██║  ██╗    ███████╗
%   ╚═╝   ╚═╝  ╚═╝╚══════╝╚═╝  ╚═╝    ╚══════╝
\Task{DualShock Controller}


\begin{qa}{Oscilloscope Measurements}
\begin{figure}[h!]
	\centering
	% \includegraphics[width=1.0\linewidth]{your filename here}
	\dummyimage
	\caption{Screenshot(s) showing the main polling command (make sure that the individual bits are visible, insert more than one image if necessary)}
\end{figure}

\begin{center}

Complete the table with the actual values shown in your measurement screenshot.
\begin{center}
\ttfamily
\begin{tabular}{|l|l|l|l|l|l|l|l|l|l|}
\hline
byte    & 1    & 2    & 3    & 4     & 5     & 6     & 7     & 8     & 9    \\ \hline
command & 0x01 & 0x42 & 0x00 & $M_S$ & $M_L$ & 0x00  & 0x00  & 0x00  & 0x00 \\ \hline
data    & 0xff & 0x73 & 0x5a & $D_1$ & $D_2$ & $X_R$ & $Y_R$ & $X_L$ & $Y_L$ \\ \hline
\end{tabular}
\end{center}
\end{center}

\end{qa}


%%%%%%%%%%%%%%%%%%%%%%%%%%%%%%%%%%%%%%%%%%%%%%%%%%%%%%%%%%%%%%%%%%%%%%%%%%%%%%%%

%████████╗ █████╗ ███████╗██╗  ██╗    ██████╗ 
%╚══██╔══╝██╔══██╗██╔════╝██║ ██╔╝    ╚════██╗
%   ██║   ███████║███████╗█████╔╝      █████╔╝
%   ██║   ██╔══██║╚════██║██╔═██╗      ╚═══██╗
%   ██║   ██║  ██║███████║██║  ██╗    ██████╔╝
%   ╚═╝   ╚═╝  ╚═╝╚══════╝╚═╝  ╚═╝    ╚═════╝ 
\Task{Bonus: SignalTap Measurement}

\begin{qa}{Trigger Condition}
	\begin{figure}[h!]
		\centering
		% \includegraphics[width=1.0\linewidth]{your filename here}
		\dummyimage
		\caption{Screenshot showing the trigger condition}
	\end{figure}
\end{qa}
%%%%%%%%%%%%%%%%%%%%%%%%%%%%%%%%%%%%%%%%%%%%%%%%%%%%%%%%%%%%%%%%%%%%%%%%%%%%%%%%

\begin{qa}{Measurement Screenshot}
	\begin{figure}[h!]
		\centering
		% \includegraphics[width=1.0\linewidth]{your filename here}
		\dummyimage
		\caption{Screenshot showing ths signal traces of one of the response time measurements}
	\end{figure}
\end{qa}
%%%%%%%%%%%%%%%%%%%%%%%%%%%%%%%%%%%%%%%%%%%%%%%%%%%%%%%%%%%%%%%%%%%%%%%%%%%%%%%%

\begin{qa}{Response Time Measurement}
\begin{center}
\begin{tabular}{|l|c|c|c|c|c|c|c|c|c|}
	\hline
	Measurement   & 1 & 2 & 3 & 4 & 5 & 6 & 7 & 8\\\hline
	Response time &   &   &   &   &   &   &   &  \\\hline
\end{tabular}
\end{center}
\end{qa}
%%%%%%%%%%%%%%%%%%%%%%%%%%%%%%%%%%%%%%%%%%%%%%%%%%%%%%%%%%%%%%%%%%%%%%%%%%%%%%%%


\begin{qa}{Why is the response time not a constant value? What do you think contributes to it?}

\end{qa}
%%%%%%%%%%%%%%%%%%%%%%%%%%%%%%%%%%%%%%%%%%%%%%%%%%%%%%%%%%%%%%%%%%%%%%%%%%%%%%%%

\end{document}
